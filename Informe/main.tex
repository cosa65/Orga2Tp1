\documentclass[spanish,a4paper]{article}

% Paquetes generales
\usepackage{ifthen}
\usepackage{amssymb}
\usepackage{amsmath}
\usepackage{multicol}
\usepackage[absolute]{textpos}
\usepackage{hyperref}
%\usepackage{graphicx}
\usepackage{caratula}


%\include{caratula}
\begin{document}

\titulo{Trabajo Pr\'{a}ctico}
\subtitulo{}

\fecha{\today}

\materia{Organizaci\'{o}n Del Computador 2}
\grupo{}

\integrante{Fosco, Martin Esteban}{449/13}{mfosco2005@yahoo.com.ar}
\integrante{Palladino, Juli\'{a}n}{231/13}{julianpalladino@hotmail.com}
\integrante{De Carli, Nicol\'{a}s}{164/13}{nikodecarli@gmail.com}

\maketitle

\section{Introducci\'{o}n}

Este Trabajo Pr\'{a}ctico se ha centrado en explorar el modelo de programaci\'{o}n \textbf{SIMD}, us\'{a}ndolo para una aplicaci\'{o}n popular del set de instrucciones SIMD de intel (SSE), procesamiento de im\'{a}genes y videos.\\
En particular, se implementaron 4 filtros de im\'{a}genes: Cropflip, Sierpinski, Bandas y Motion Blur.\\

Se ha buscado aprovechar los beneficios de SSE: 

\begin{itemize}

	\item Procesar conjuntos de datos de manera eficiente, ejecutando de manera paralela (al mismo tiempo) la misma instrucci\'{o}n sobre distintos datos.
	
	\item El uso de los registros XMM, los cuales proveen una gran versatilidad con la opci\'{o}n de ejecutar operaciones de punto flotante y enteras con distintas precisiones (sobre datos empaquetados o escalares).

	\item Minimizar los accesos a memoria.
	
	\item No se me ocurri\'{o} nada mas, pero llenen o borren, como les parezca mejor :)

\end{itemize} 
Luego de implementar en C y asm (procesando de manera escalar y vectorial los datos, respectivamente) los filtros de im\'{a}genes se ha comparado la performance de ambos modelos de programaci\'{o}n para determinar de manera aproximada la ventaja que se gana al trabajar sobre muchos datos de manera simult\'{a}nea.

\newpage

\section{Desarrollo}


\newpage
\section{Conclusi\'{o}n}

\end{document}
